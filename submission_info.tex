\documentclass[12pt]{article}
\usepackage[margin=1in]{geometry}
\usepackage{mathptmx} % Times-like font

\title{Required Supplementary Information}
\date{}
\author{}

\begin{document}

\maketitle

\section*{Content Category Statement}
This article primarily addresses the **development of an innovative technique useful to those in preservation practice**. It introduces accessible machine learning workflows—specifically Factor Analysis and Feature Importance—that allow practitioners to translate complex, multi-variable datasets into actionable preservation priorities. Additionally, it supports the **development of a new concept** in preservation philosophy by demonstrating a shift from purely intuition-based assessment to data-driven decision-making for managing large-scale heritage portfolios.

\section*{Author Contributions}
Joe Kallas and Rebecca Napolitano contributed equally to all portions of this project.

\section*{Prior Publication Statement}
This article is original work and has not been previously published, nor is it under consideration for publication elsewhere.

\section*{Teaser}
Discover how machine learning transforms complex heritage data into clear, actionable preservation priorities with this step-by-step guide to factor analysis and feature importance.

\section*{Author Biographies and Contact Information}

\textbf{1. Rebecca Napolitano} \\
\textit{Affiliation:} Penn State University, College of Engineering, University Park, United States, PA, 16802 \\
\textit{Email:} nap@psu.edu \\
\textbf{Bio:} Dr. Rebecca Napolitano, Assistant Professor of Architectural Engineering at Penn State, leverages machine learning to create more efficient and accurate methods for documenting and diagnosing heritage buildings. By automating analysis of images and 3D data, her research helps to better monitor, understand, and safeguard irreplaceable heritage sites around the world.

\vspace{1em}

\textbf{2. Joe Kallas} \\
\textit{Affiliation:} Penn State University, College of Engineering, University Park, United States, PA, 16802 \\
\textit{Email:} jkallas@psu.edu \\
\textbf{Bio:} Joe Kallas is an architect and cultural heritage expert contributing to UNESCO-led recovery efforts in crisis-affected regions worldwide, including Beirut, Ukraine, and Syria. His work integrates AI, 3D assessment, and digital tools to support local communities in documenting damage, planning emergency responses, and protecting historic urban environments under threat.

\end{document}
