
\section{Illustrative Scenarios in Interpretation}

[JOE: CAN YOU TRY TO MAKE SOMETHING FOR THIS? THE REVIEWER WANTS EXAMPLES OF GOOD RESULTS, POOR RESULTS, AND UNCERTAIN RESULTS TO SHOW POTENTIAL AND LIMITATIONS.]

To illustrate interpretation in practice, consider three scenarios encountered during validation:

\begin{itemize}
    \item \textbf{Scenario A: Convergent Evidence.} The model identified "rising damp" as the top predictor for a block of masonry buildings. Field inspection confirmed consistent salt efflorescence at the base, validating the model's prioritization.
    
    \item \textbf{Scenario B: Model Failure.} A building rated as "low risk" by the model exhibited severe structural cracking. Investigation revealed the cracks were caused by a recent vehicle impact—a stochastic event not captured in the historical environmental dataset. This highlights the need for continuous data updates.
    
    \item \textbf{Scenario C: High Uncertainty.} For a group of timber structures, the model predicted moderate risk with high variance. Field assessment showed micro-climates (shading from new construction) were driving decay, a variable missing from the original dataset. This prompted the installation of localized sensors to refine future models.
\end{itemize}
